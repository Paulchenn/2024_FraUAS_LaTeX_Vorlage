\newcommand{\myName}{Max Mustermann}
\newcommand{\ThesisTitle}{Titel}
\newcommand{\ThesisTitleShort}{Kurztitel}
\newcommand{\ThesisSubtitle}{Untertitel}
\newcommand{\ThesisDegree}{Bachelor of Engineering (B.Eng.)} 
%\newcommand{\ThesisDegree}{Master of Science (M. Sc.)}
\newcommand{\myStudentId}{123456}
\newcommand{\Supervisor}{Donald Duck}
\newcommand{\CoSupervisor}{Dagobert Duck}
\newcommand{\CompSupervisor}{Trick Tick Track}
\newcommand{\Faculty}{Faculty of Computer Science and Engineering}
\newcommand{\University}{Frankfurt University of Applied Sciences}
\newcommand{\UniversityLocation}{Frankfurt}
\newcommand{\ThesisDeliveryDate}{25. Februar 2025}
\newcommand{\CompanyName}{NoCompanyInvolved}  % don't modify this line, if no company is involved

%\renewcommand{\theenumi}{\textbf{(\alph*{enumi})}} % Setze die Aufzählung auf fett


\newcommand{\VarPicWidthA}{\textwidth}
\newcommand{\VarPicWidthB}{0.9\textwidth}
\newcommand{\VarPicWidthC}{0.8\textwidth}
\newcommand{\VarPicWidthD}{0.7\textwidth}
\newcommand{\VarPicWidthE}{0.6\textwidth}
\newcommand{\VarPicWidthF}{0.49\textwidth}
\newcommand{\VarPicWidthG}{0.32\textwidth}
\newcommand{\VarPicWidthH}{0.24\textwidth}
\newcommand{\VarPicWidthI}{0.2\textwidth}
\newcommand{\VarPicWidthJ}{0.469\textwidth}
\newcommand{\VarPicWidthK}{0.45\textwidth}

%\definecolor{blue40} {rgb}{0.0000, 0.3843, 0.6039}
\definecolor{blue50} {rgb}{0.0000, 0.4824, 0.7529}
%\definecolor{blue55} {rgb}{0.0000, 0.5333, 0.8314}
\definecolor{turq50} {rgb}{0.0941, 0.5137, 0.4941}
%\definecolor{green40}{rgb}{0.0000, 0.4235, 0.2275}
\definecolor{green50}{rgb}{0.0000, 0.5333, 0.2902}
%\definecolor{green55}{rgb}{0.1294, 0.5843, 0.3412}
\definecolor{purp40} {rgb}{0.6196, 0.1569, 0.5882}
%\definecolor{red40}  {rgb}{0.7451, 0.0000, 0.0157}
\definecolor{red50}  {rgb}{0.9294, 0.0000, 0.0725}
%\definecolor{red55}  {rgb}{1.0000, 0.1294, 0.1412}
\definecolor{gray40} {rgb}{0.3490, 0.3686, 0.3843}%
\definecolor{gray70} {rgb}{0.6431, 0.6706, 0.7020}%

\newcommand{\ti}[1]{_\mathrm{#1}}
\newcommand\tab[1][1cm]{\hspace*{#1}}

\newcommand{\abs}[1]{\ensuremath{\left\vert#1\right\vert}}
\newcommand{\rdown}[1]{\ensuremath{\left\lfloor#1\right\rfloor}}
%\DeclarePairedDelimiter\abs{\lvert}{\rvert}

% Abstand für den Rahmen auf 0pt setzen
\setlength{\fboxsep}{0pt}

\newlist{listNorm}{enumerate}{4}
\setlist[listNorm,1]{label={\scriptsize\(\bullet\)}}
\setlist[listNorm,2]{label={\scriptsize-}}
\newlist{listArab}{enumerate}{4}
\setlist[listArab]{label*=\arabic*.}
\newlist{listAlph}{enumerate}{4}
\setlist[listAlph]{label*=(\alph*)}
\newlist{listAlphB}{enumerate}{4}
\setlist[listAlphB]{label*=\textbf{(\alph*)}}

% Definieren von Stilen für die Formen und Pfeile im Diagramm
\tikzstyle{startstop} 		= [rectangle, rounded corners, minimum width=3cm, minimum height=1cm,text centered, draw=black]
\tikzstyle{process}			= [rectangle, rounded corners, text centered, draw=black]
\tikzstyle{info} 			= [rectangle, text centered, draw=none]
\tikzstyle{decision} 		= [diamond, minimum width=3cm, minimum height=1cm, text centered, draw=black]
\tikzstyle{arrow} 			= [thick,->,>=stealth]
\tikzstyle{arrow2} 			= [thick,<->,>=stealth]
\tikzstyle{vorangeganen} 	= [rectangle, rounded corners, text centered, draw=black, fill=blue55, fill opacity=0.1]
\tikzstyle{hier} 			= [rectangle, rounded corners, text centered, draw=black]
\tikzstyle{weiterführen} 	= [rectangle, rounded corners, text centered, draw=black, fill=green55, fill opacity=0.1]


%%%%%%%%%%%%%%%%%%%%% FOR ENGLISH LANGUAGE THESIS: %%%%%%%%%%%%%%%%%%%%

% By default the thesis language is german
% if you want to set it to ENGLISH, then UNCOMMENT the FOLLOWING LINE by removing the leading "%":

%\newcommand*{\ThesisLanguageIsEnglish}{}

%% End of File